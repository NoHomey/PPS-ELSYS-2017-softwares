\documentclass[12pt]{article}

\usepackage[left=3cm,right=3cm,top=1cm,bottom=2cm]{geometry}
\usepackage{amsmath,amsthm}
\usepackage{amssymb}
\usepackage{tikz}
\usepackage{lipsum}
\usepackage[T1,T2A]{fontenc}
\usepackage[utf8]{inputenc}
\usepackage[bulgarian]{babel}
\usepackage[normalem]{ulem}

\newcommand{\N}{\mathbb{N}}
\newcommand{\R}{\mathbb{R}}

\setlength{\parindent}{0mm}
    
\title{Вектори и матрици}
\author{Иво Стратев}
    
\begin{document}
\maketitle

\section*{Вектори}

\subsection*{Наредена двойка от реални числа}

Произволна наредена двойка от реални числа записваме като: $(a, \; b)$, където $a, \; b \in \R$. \\

Множеството от всички наредени двойки от реални числа означаваме с $\R^2$ и на езика на математиакта записваме като:
$\R^2 = \{(a, \; b) \; | \; a, \; b \in \R\}$ \\

Подобно на начина, по който си мислим за реални числа като точки от една безкрайна права множеството $\R^2$ можем
да си мислим че е множеството от точки в равнината. По този начин на наредената двойка $(1, \; 2)$ можем да съпоставим
точка от равнината с координати $(1, \; 2)$, която се намира на разтояние $1$ спрямо центъра на координаттната система
по абцисата и на разтояние $2$ по ординатата . Графично: \\\\

\begin{tikzpicture}
    \draw[thin,gray!40] (-4,-4) grid (4,4);
    \draw[<->] (-4,0)--(4,0) node[right]{$x$};
    \draw[<->] (0,-4)--(0,4) node[above]{$y$};
    \draw[fill] (1, 2) circle (2pt) node[anchor=south west]{$(1, \; 2)$};
  \end{tikzpicture}

\subsection*{Наредена тройка от реални числа}
  
Аналогично на понятието наредена двойка от реални числа можем да си въведем понятието наредена тройка от реални числа.

Произволна наредена тройка от реални числа записваме като: $(a, \; b, \; c)$, където $a, \; b, \; c \in \R$. \\
  
Множеството от всички наредени тройки от реални числа означаваме с $\R^3$ и на езика на математиакта записваме като:
$\R^3 = \{(a, \; b, \; c) \; | \; a, \; b, \; c \in \R\}$ \\
  
Подобно на начина, по който си мислим за наредените двойки от реални числа катоточки в равнината.
Можем да си мислим за една наредена тройка от наредени реални числа като точка от пространството.
Така на произволна наредена тройка $(x, \; y, \; z)$ от реални числа можем по единствен начин да съпоставим
точка от пространството, която се намира на отместване $x$ спрямо началото на координатната система по абсцисата (остта $Ox$),
отместване $y$ спрямо началото на координатната система по ординатата (остта $Oy$) и на отместване $z$ по апликатата (остта $Oz$).

\subsection*{Наредена n-торка от реални числа}

Понятието наредена n-торка от реални числа можем да изградим като обобщим това за наредена двойка, тройка. \\

Произволна наредена n-торка от реални числа записваме като: $(a_1, \; a_2, \; \dots, \; a_n)$, където $a_1, \; a_2, \; \dots, \; a_n \in \R$. \\

Аналогично множеството от всички наредени n-торки от реални числа ще обозначаваме с $\R^n$ и
$\R^n = \{(a_1, \; a_2, \; \dots, \; a_n) \; | \; a_1, \; a_2, \; \dots, \; a_n \in \R\}$. \\

При $n > 3$ вече нямаме пряка геометрична репрезентация ... \\

\subsection*{Вектор}

Нека $n$ е фиксирано естествено число (тоест $n \in \N$). \\\\

Сега ако в множеството $\R^n$ си дефинираме по подходящ начин операция събиране на наредени n-торки и умножение с реално число на наредена n-торка получаваме нещо, на което в математиката (а и не само там)
наричаме вектор(и). Тези две операции дефинираме по следния начин: \\

Нека $a = (a_1, \; a_2, \; \dots, \; a_n)$ и $b = (b_1, \; b_2, \; \dots, \; b_n)$ са произволни наредени n-торки от реални числа. Тогава
$a + b \overset{def}{=} (a_1 + b_1, \; a_2 + b_2, \; \dots, \; a_n + b_n)$. Тоест извършваме покомпонентно операцията събиране на реални числа за всяка компонента. \\

Нека $a = (a_1, \; a_2, \; \dots, \; a_n)$ е произволна наредена n-торка от реални числа и $\lambda$ е произволно реално число. Тогава
$\lambda.a \overset{def}{=} (\lambda.a_1, \; \lambda.a_2, \; \dots, \; \lambda.a_n)$. Отново извършваме покомпонентно операцията умножение на реални числа за всяка компонента. \\

Под вектор ние ще разбираме всяка наредена n-торка от реални числа имайки в предвид двете операции, които дефинирахме.\\

Забележка: по начина по-който дефинирахме операцията умножение с число е ясно, че можем да я прилагаме върху каквато и да е
наредена k-торка (тоес и за наредена двойка, тройка, четворка и тн.). Обаче операцията събиране можем да прилагаме само върху
вектори от един и същ тип (тоест не можем директно да съберем наредена двойка с тройка). \\

Забележка: честно ще изпускаме да пишем точката за умножение подобно на изпускането ѝ което сте свикнали да правите в часовете по математика. \\

Забележка: понятието вектор в математика представлява абстракция която има широко приложение и различни математически обекти се превръщат във вектори с подходяща
дефиниция на двете операции. В този курс ще се занимаваме с доста ограничени видове вектори, които обаче се приемат за фундаментални и основни и напрактика почти
всички вектори са подобни на тях.

\subsubsection*{Примери:}

$(1, \; 2, \; 3, \; 4) + (2, \; 4, \; 6, \; 8) = (1 + 2, \; 2 + 4, \; 3 + 6, \; 4 + 8) = (3, \; 6, \; 9, \; 12)$ \\\\

$3(1, \; 2, \; 3, \; 4) = (3.1, \; 3.2, \; 3.3, \; 3.4) = (3, \; 6, \; 9, \; 12)$

\subsubsection*{Геометрична интерпретация на двете операции в множеството $\R^2$}

Геометрично векторите ще изобразяваме като вместо точка свържем центъра на координатната система и за край изпозлваме стрелка в съответната посока. Тоест:\\\\

\begin{tikzpicture}
    \draw[thin,gray!40] (-4,-4) grid (4,4);
    \draw[<->] (-4,0)--(4,0) node[right]{$x$};
    \draw[<->] (0,-4)--(0,4) node[above]{$y$};
    \draw[fill] (1, 2) circle (2pt) node[anchor=south west]{$(1, \; 2)$};
    \draw[line width=2pt,green,-stealth](0,0)--(1,2);
\end{tikzpicture} \\\\
  
Нека сега съберем векторите: \\

\begin{tikzpicture}
    \draw[thin,gray!40] (-5,-5) grid (5,5);
    \draw[<->] (-5,0)--(5,0) node[right]{$x$};
    \draw[<->] (0,-5)--(0,5) node[above]{$y$};
    \draw[line width=2pt,blue,-stealth](0,0)--(1,3) node[anchor=south west]{$\boldsymbol{(1,3)}$};
    \draw[line width=2pt,red,-stealth](0,0)--(2,1) node[anchor=south west]{$\boldsymbol{(2,1)}$};
    \draw[line width=2pt,purple,-stealth](0,0)--(3,4) node[anchor=south west]{$\boldsymbol{(3,4)}$};
\end{tikzpicture} \\\\

Казваме, че събираме векторите е по така нареченото правило на триъгълника.
Тоест като в края на първия успоредно пренесем втория такаче началото на пренесения на съвпада с края на другия. Или: \\

\begin{tikzpicture}
    \draw[thin,gray!40] (-5,-5) grid (5,5);
    \draw[<->] (-5,0)--(5,0) node[right]{$x$};
    \draw[<->] (0,-5)--(0,5) node[above]{$y$};
    \draw[line width=2pt,blue,-stealth](0,0)--(1,3) node[anchor=south west]{$\boldsymbol{(1,3)}$};
    \draw[line width=2pt,red,-stealth](0,0)--(2,1) node[anchor=south west]{$\boldsymbol{(2,1)}$};
    \draw[line width=2pt,purple,-stealth](0,0)--(3,4) node[anchor=south west]{$\boldsymbol{(3,4)}$};
    \draw[line width=2pt,blue,dashed,-stealth](2,1)--(3,4);
\end{tikzpicture} \\\\

Или казваме, че събираме вектори по правилото на успоредника, защото всъщност сборът на двата вектора е диагонал
на успоредник построен със "страни" двата вектора, или: \\

\begin{tikzpicture}
    \draw[thin,gray!40] (-5,-5) grid (5,5);
    \draw[<->] (-5,0)--(5,0) node[right]{$x$};
    \draw[<->] (0,-5)--(0,5) node[above]{$y$};
    \draw[line width=2pt,blue,-stealth](0,0)--(1,3) node[anchor=south east]{$\boldsymbol{(1,3)}$};
    \draw[line width=2pt,red,-stealth](0,0)--(2,1) node[anchor=south west]{$\boldsymbol{(2,1)}$};
    \draw[line width=2pt,purple,-stealth](0,0)--(3,4) node[anchor=north west]{$\boldsymbol{(3,4)}$};
    \draw[line width=2pt,blue,dashed,-stealth](2,1)--(3,4);
    \draw[line width=2pt,red,dashed,-stealth](1,3)--(3,4);
\end{tikzpicture} \\\\

Нека сега да умножим вектора $(1, \; 2)$ с числото $2$ и с числото $-1$.\\

\begin{tikzpicture}
    \draw[thin,gray!40] (-5,-5) grid (5,5);
    \draw[<->] (-5,0)--(5,0) node[right]{$x$};
    \draw[<->] (0,-5)--(0,5) node[above]{$y$};
    \draw[line width=2pt,blue,-stealth](0,0)--(1,2) node[anchor=south east]{$\boldsymbol{(1,2)}$};
    \draw[line width=2pt,red,-stealth](0,0)--(-1,-2) node[anchor=south east]{$\boldsymbol{(-1,-2)}$};
    \draw[line width=2pt,green,dashed,-stealth](0,0)--(2,4) node[anchor=south east]{$\boldsymbol{(2,4)}$};
\end{tikzpicture} \\\\

Очевидно всеки вектор геометрически се характеризира с посока и дължина. Сега ще получим формула за дължината на даден вектор,
която след това ще обобщим за n-мерните вектори (наредните n-торки). \\

Нека си начертаем вектора $(1, \; 2)$, след което успоредно си пренесем разтоянието по ординатата в края на разтоянието по абсцисата: \\

\begin{tikzpicture}
    \draw[thin,gray!40] (-4,-4) grid (4,4);
    \draw[<->] (-4,0)--(4,0) node[right]{$x$};
    \draw[<->] (0,-4)--(0,4) node[above]{$y$};
    \draw[line width=2pt,blue,-stealth](0,0)--(1,2) node[anchor=south east]{$\boldsymbol{(1,2)}$};
    \draw[line width=1pt,blue,dashed,-stealth](1,0)--(1,2);
    \draw[line width=1pt,blue,dashed,-stealth](0,0)--(1,0);
\end{tikzpicture} \\\\

Очевидно получаваме правоъгален триъгълник с дължини на катетите $1$ и $2$ тогава от формулата на Питагор
за хипотенузата на този тригълник $\rho$ получаваме: \\
$\rho^2 = 1^2 + 2^2$, но понеже дължината винаги е неотриателно число получаваме: \\
$\rho = \sqrt{1^2 + 2^2} = \sqrt{5}$. \\

Нека сега сметнем дължината $h$ на вектора $2(1, \; 2) = (2, \; 4)$: \\
$h^2 = \sqrt{2^2 + 4^2} = \sqrt{2^2.1^2 + 2^2.2.^2} = \sqrt{2^2.(1^2 + 2^2)} = 2\sqrt{5}$ \\

Ако $a = (a_1, \; a_2)$ дължината на вектора $a$ ще означаваме с $\|a\| = \sqrt{a_1^2 + a_2^2}$ \\

Сега ще си докажем едно много важно свойстово на умножението на вектор с число: \\

Нека $a = (x, \; y) \in \R^2$ и $\lambda \in \R$ тогава $\|\lambda . a\| = |\lambda|.\|a\|$ \\

Доказателство: $\|\lambda . a\| = \sqrt{(\lambda x)^2 + (\lambda y)^2} = \sqrt{\lambda^2(x^2 + y^2)} = \\
= \sqrt{\lambda^2}\sqrt{x^2 + y^2} = |\lambda|.\|a\| \qed$ \\

Очевидно посоката на даден вектор можем да определим на базата ъгъла, който той сключва с абсцисата.
На помощ ни идват любимите на учениците тригонометрични функции $\sin(x)$ и $\cos(x)$.
Ще използваме горния чертеж в комбинация със стандартните означения за триъгълници: \\

\begin{tikzpicture}
    \draw[thin,gray!40] (-4,-4) grid (4,4);
    \draw[<->] (-4,0)--(4,0) node[right]{$x$};
    \draw[<->] (0,-4)--(0,4) node[above]{$y$};
    \draw[line width=2pt,blue,-stealth](0,0)--(1,2) node[anchor=south east]{$\boldsymbol{(1,2)}$};
    \draw[line width=1pt,blue,dashed,-stealth](1,0)--(1,2);
    \draw[line width=1pt,blue,dashed,-stealth](0,0)--(1,0);
    \draw (0.5, 0) node[anchor=north]{$a$};
    \draw (1, 1) node[anchor=west]{$b$};
    \draw (0.5, 1) node[anchor=south east]{$c$};
    \draw (0.1, 0) node[anchor=south west]{$\varphi$};
\end{tikzpicture} \\\\

Очевидно $a = 1, \; b = 2, \; c = \|(1, \; 2)\| = \sqrt{5}$ \\

Така $ \sin(\varphi) \overset{def}{=} \frac{b}{c} = \frac{2\sqrt{5}}{5} $ и $ \cos(\varphi) \overset{def}{=} \frac{a}{c} = \frac{\sqrt{5}}{5} $. Нека сега се уверим,
че умножавайки вектора $(1, \; 2)$ с произволно положително реално число реално резултатния вектор ще има същата посока.
Нека $\mu \in \R$ и $\mu > 0$ тогава за $\mu(1, \; 2) = (\mu, \; 2\mu) $ имаме
$ \sin(\psi) \overset{def}{=} \frac{\mu.b}{\mu.c} = \frac{\mu2\sqrt{5}}{\mu5} = \frac{2\sqrt{5}}{5} = \sin(\varphi) $
и $ \cos(\psi) \overset{def}{=} \frac{\mu.a}{\mu.c} = \frac{\mu\sqrt{5}}{\mu5} = \frac{\sqrt{5}}{5} = \cos(\varphi) $

Нека сега $\mu < 0 \implies |\mu| = -\mu \implies \mu = -|\mu| $. \\
Тогава за $\mu(1, \; 2) = (\mu, \; 2\mu) $ имаме: \\
$ \sin(\psi) \overset{def}{=} \frac{-|\mu|.b}{|\mu|.c} = \frac{-|\mu|2\sqrt{5}}{|\mu|5} = -\frac{2\sqrt{5}}{5} = -\sin(\varphi) $ \\
и $ \cos(\psi) \overset{def}{=} \frac{-|\mu|.a}{|\mu|.c} = \frac{-|\mu|\sqrt{5}}{|\mu|5} = -\frac{\sqrt{5}}{5} = -\cos(\varphi) $ \\

Очевидно умножавайки вектор с отрицателно реално число получаваме вектор с противоположна посока. \\

Ето защо когато умножаваме даден вектор с число още казваме, че го умножаваме със скалар или че скалираме дадения вектор.
Защото по този начин променяме само дължината на дадения вектор, но неговата посока се запазва ако умножаваме с положително по знак число и
получаваме противоположна посока в случай на отрицателно число. \\

Знаем, че модула на реално число играе ролята на разтояние от центъра на реалната права, токчата отговаряща на числото $0$.
Също така знаем, че ако $\lambda \in \R$ то $|\lambda| = \sqrt{\lambda^2}$ И има точно две реални числа на разтояние
$|\lambda|$ - $\lambda$ и $-\lambda$, защото върху една права имаме точно две посоки положителна и отрицателна. \\

В равнината обаче имаме безброй много посоки защото имаме безборй много ъгли, които дават различни двойка стойности $(\sin(\varphi), \; \cos(\varphi))$,
по-точно $\varphi \in [0, \pi)$, но това е отворен интервал от реални числа и с помощта на математиката (логиката и теорията на множествата) лесно
се доказва, че интервала $[0, \pi)$ съдържа безброй много реални числа (всъщност той съдържа толкова на брой реални числа колкото и самото множество $\R$). \\

Така получаваме цяла окръжност от точки (вектори), които са на едно и също разстояние: \\

\begin{tikzpicture}
    \draw[thin,gray!40] (-4,-4) grid (4,4);
    \draw[<->] (-4,0)--(4,0) node[right]{$x$};
    \draw[<->] (0,-4)--(0,4) node[above]{$y$};
    \draw[line width=2pt,blue,-stealth](0,0)--(1,2) node[anchor=south west]{$\boldsymbol{(1,2)}$};
    \draw[line width=2pt,blue,dashed,-stealth](0,0) circle ({sqrt(5)});
\end{tikzpicture}

\subsubsection*{Дефиниция за дължина на вектор}

Дължина и разстояние за нас практически са едно и също понятие и така
дължина на едномерен вектор всъщност съвпада с модула на числото, коеот е единствена компонента на вектора, тоест:\\

$\|(a)\| = \sqrt{a^2} = |a|$. \\

Двумерния случай вече е ясен, дължина там е мярката на радиоса на окръжността, която можем да опишем около даден вектор. \\

В тримерния случай по аналогия на идеята с окръжността получаваме, че дължина на тримерен вектор всъщност
ще съвпада с радиоса на единствената сфера, която можем да опишем около дадения вектор, тоест: \\

$\|(a, \; b, \; c)\| = \sqrt{a^2 + b^2 + c^2}$. \\

Така за n-мерен вектор е естесвено да стигнем до следната формула: \\

$\|(a_1, \; a_2, \; \dots, \; a_n)\| = \sqrt{a_1^2 + a_2^2 + \dots + a_n^2} = \displaystyle\sqrt{\displaystyle\sum_{i = 1}^n a_i^2}$ \\

Очевидно отново ще имаме свойстовото $\|\lambda.(a_1, \; a_2, \; \dots, \; a_n)\| = |\lambda|.\|(a_1, \; a_2, \; \dots, \; a_n)\|$

\section*{Матрици}

Забележка: В този курс ще разглеждаме само матрици от реални числа. \\

Неформалната ни дефиниция за матрица с m реда и n стълба е: правоъгълна табличка от числа (с размерност $m \times n$). \\

Множеството от всички матрици с m реда и n стълба от рални числа ще бележим с $\R_{m \times n}$ и то за нас ще бъде следното множество: \\\\

$ \R_{m \times n} = \left\{
    \begin{pmatrix}
        a_{11} & a_{12} & a_{13} & \dots  & a_{1n} \\
        a_{21} & a_{22} & a_{23} & \dots  & a_{2n} \\
        \vdots & \vdots & \vdots & \ddots & \vdots \\
        a_{m1} & a_{m2} & a_{m3} & \dots  & a_{mn}
    \end{pmatrix} \; \Bigg| \; i = 1, \; \dots, \; m, \; j = 1, \; \dots, \; n \quad a_{ij} \in \R
\right\} $ \\\\

\subsection*{Да направим Матриците вектори}

Първото нещо, което забелязваме е че можем да "постройм" функция, която на всяка матрица от $\R_{m \times n}$ съпоставим точно един
елемент на множеството $\R^{m.n}$. Една такава функция на езика на математиката е например следната: \pagebreak

$
    \varphi \; : \; \R_{m \times n} \to \R^{m.n} \\\\
    \begin{pmatrix}
        a_{11} & a_{12} & a_{13} & \dots  & a_{1n} \\
        a_{21} & a_{22} & a_{23} & \dots  & a_{2n} \\
        \vdots & \vdots & \vdots & \ddots & \vdots \\
        a_{m1} & a_{m2} & a_{m3} & \dots  & a_{mn}
    \end{pmatrix} \mapsto (a_{11}, \; \dots, \; a_{1n}, \; a_{21}, \; \dots, \; a_{2n}, \dots, \; a_{m1}, \;  \dots, \; a_{mn})
$ \\\\

Много ни се иска матриците да се държат като вектори, за целта искаме да си дефинирами операции събиране на матрци и умножение на матрица с число.
Тоест много ни се иска да е изпълнено следното свойство: \\\\

Ако $A = \begin{pmatrix}
    a_{11} & a_{12} & a_{13} & \dots  & a_{1n} \\
    a_{21} & a_{22} & a_{23} & \dots  & a_{2n} \\
    \vdots & \vdots & \vdots & \ddots & \vdots \\
    a_{m1} & a_{m2} & a_{m3} & \dots  & a_{mn}
\end{pmatrix}, \; B = \begin{pmatrix}
    b_{11} & b_{12} & b_{13} & \dots  & b_{1n} \\
    b_{21} & b_{22} & b_{23} & \dots  & b_{2n} \\
    \vdots & \vdots & \vdots & \ddots & \vdots \\
    b_{m1} & b_{m2} & b_{m3} & \dots  & b_{mn}
\end{pmatrix}$ и $\lambda \in \R$\\\\

да е изпълнено: \\
$\varphi(A + B) = \varphi(A) + \varphi(B)$ (1) \\
$\varphi(\lambda.A) = \lambda.\varphi(A)$ (2) \\

От всичко, което знаeм до сега последователно получаваме: \\

$\varphi(A) + \varphi(B) = \\\\
= \varphi\left(\begin{pmatrix}
    a_{11} & a_{12} & a_{13} & \dots  & a_{1n} \\
    a_{21} & a_{22} & a_{23} & \dots  & a_{2n} \\
    \vdots & \vdots & \vdots & \ddots & \vdots \\
    a_{m1} & a_{m2} & a_{m3} & \dots  & a_{mn}
\end{pmatrix}\right) + \varphi\left(\begin{pmatrix}
    b_{11} & b_{12} & b_{13} & \dots  & b_{1n} \\
    b_{21} & b_{22} & b_{23} & \dots  & b_{2n} \\
    \vdots & \vdots & \vdots & \ddots & \vdots \\
    b_{m1} & b_{m2} & b_{m3} & \dots  & b_{mn}
\end{pmatrix}\right) = \\\\
= (a_{11}, \; \dots, \; a_{1n}, \; a_{21}, \; \dots, \; a_{2n}, \dots, \; a_{m1}, \;  \dots, \; a_{mn}) \\
+ (b_{11}, \; \dots, \; b_{1n}, \; b_{21}, \; \dots, \; b_{2n}, \dots, \; b_{m1}, \;  \dots, \; b_{mn}) = \\ 
= (a_{11} + b_{11}, \; \dots, \; a_{1n} + b_{1n}, \; a_{21} + b_{21}, \; \dots, \; a_{2n} + b_{2n}, \dots, \; a_{m1} + b_{m1}, \;  \dots, \; a_{mn} + b_{mn}) = \\\\
= \varphi\left(\begin{pmatrix}
    a_{11} + b_{11} & a_{12} + b_{12} & a_{13} + b_{13} & \dots  & a_{1n} + b_{1n} \\
    a_{21} + b_{21} & a_{22} + b_{22} & a_{23} + b_{23} & \dots  & a_{2n} + b_{2n} \\
    \vdots & \vdots & \vdots & \ddots & \vdots \\
    a_{m1} + b_{m1} & a_{m2} + b_{m2} & a_{m3} + b_{m3} & \dots  & a_{mn} + b_{mn}
\end{pmatrix}\right) = \varphi(A + B) \implies \\\\\\
A + B = \\\\
= \begin{pmatrix}
    a_{11} & a_{12} & a_{13} & \dots  & a_{1n} \\
    a_{21} & a_{22} & a_{23} & \dots  & a_{2n} \\
    \vdots & \vdots & \vdots & \ddots & \vdots \\
    a_{m1} & a_{m2} & a_{m3} & \dots  & a_{mn}
\end{pmatrix} + \begin{pmatrix}
    b_{11} & b_{12} & b_{13} & \dots  & b_{1n} \\
    b_{21} & b_{22} & b_{23} & \dots  & b_{2n} \\
    \vdots & \vdots & \vdots & \ddots & \vdots \\
    b_{m1} & b_{m2} & b_{m3} & \dots  & b_{mn}
\end{pmatrix} = \\\\
= \begin{pmatrix}
    a_{11} + b_{11} & a_{12} + b_{12} & a_{13} + b_{13} & \dots  & a_{1n} + b_{1n} \\
    a_{21} + b_{21} & a_{22} + b_{22} & a_{23} + b_{23} & \dots  & a_{2n} + b_{2n} \\
    \vdots & \vdots & \vdots & \ddots & \vdots \\
    a_{m1} + b_{m1} & a_{m2} + b_{m2} & a_{m3} + b_{m3} & \dots  & a_{mn} + b_{mn}
\end{pmatrix}$ \\

и \\

$\lambda.\varphi(A) = \\\\
= \lambda.\varphi\left(\begin{pmatrix}
    a_{11} & a_{12} & a_{13} & \dots  & a_{1n} \\
    a_{21} & a_{22} & a_{23} & \dots  & a_{2n} \\
    \vdots & \vdots & \vdots & \ddots & \vdots \\
    a_{m1} & a_{m2} & a_{m3} & \dots  & a_{mn}
\end{pmatrix}\right) = \\\\
= \lambda.(a_{11}, \; \dots, \; a_{1n}, \; a_{21}, \; \dots, \; a_{2n}, \dots, \; a_{m1}, \;  \dots, \; a_{mn}) = \\
= (\lambda.a_{11}, \; \dots, \; \lambda.a_{1n}, \; \lambda.a_{21}, \; \dots, \; \lambda.a_{2n}, \dots, \; \lambda.a_{m1}, \;  \dots, \; \lambda.a_{mn}) = \\\\
= \varphi\left(\begin{pmatrix}
    \lambda.a_{11} & \lambda.a_{12} & \lambda.a_{13} & \dots  & \lambda.a_{1n} \\
    \lambda.a_{21} & \lambda.a_{22} & \lambda.a_{23} & \dots  & \lambda.a_{2n} \\
    \vdots & \vdots & \vdots & \ddots & \vdots \\
    \lambda.a_{m1} & \lambda.a_{m2} & \lambda.a_{m3} & \dots  & \lambda.a_{mn}
\end{pmatrix}\right) = \varphi(\lambda.A) \implies \\\\\\
\lambda.A = \lambda.\begin{pmatrix}
    a_{11} & a_{12} & a_{13} & \dots  & a_{1n} \\
    a_{21} & a_{22} & a_{23} & \dots  & a_{2n} \\
    \vdots & \vdots & \vdots & \ddots & \vdots \\
    a_{m1} & a_{m2} & a_{m3} & \dots  & a_{mn}
\end{pmatrix} = \begin{pmatrix}
    \lambda.a_{11} & \lambda.a_{12} & \lambda.a_{13} & \dots  & \lambda.a_{1n} \\
    \lambda.a_{21} & \lambda.a_{22} & \lambda.a_{23} & \dots  & \lambda.a_{2n} \\
    \vdots & \vdots & \vdots & \ddots & \vdots \\
    \lambda.a_{m1} & \lambda.a_{m2} & \lambda.a_{m3} & \dots  & \lambda.a_{mn}
\end{pmatrix} $ \\\\

Тоест операциите отново си ги дефинираме като покомпонентно извършване на съответните операции между реални числа.

\subsubsection*{Примери:}

$
\begin{pmatrix}
    1 & 3 & 0 & 7 \\
    2 & -5 & 4 & -3 \\
    1 & 0 & -1 & -1
\end{pmatrix} + \begin{pmatrix}
    4 & 3 & 0 & 1 \\
    2 & -3 & -4 & 0 \\
    6 & 0 & 0 & 1
\end{pmatrix} = \begin{pmatrix}
    1 + 4 & 3 + 3 & 0 + 0 & 7 + 1 \\
    2 + 2 & -5 - 3 & 4 - 4 & -3 + 0 \\
    1 + 6 & 0 + 0 & -1 + 0 & -1 + 1
\end{pmatrix} = \begin{pmatrix}
    5 & 6 & 0 & 8 \\
    4 & -8 & 0 & -3 \\
    7 & 0 & -1 & 0
\end{pmatrix} \\\\\\
9.\begin{pmatrix}
    1 & 3 & 0 & 7 \\
    2 & -5 & 4 & -3 \\
    1 & 0 & -1 & -1
\end{pmatrix} = \begin{pmatrix}
    9.1 & 9.3 & 9.0 & 9.7 \\
    9.2 & 9.-5 & 9.4 & 9.-3 \\
    9.1 & 9.0 & 9.-1 & 9.-1
\end{pmatrix} = \begin{pmatrix}
    9 & 27 & 0 & 63 \\
    18 & -45 & 36 & -27 \\
    9 & 0 & -9 & -9
\end{pmatrix}
$

\subsection*{Транспониране на матрица}

Нека имаме следната функция дефинирана за всяко $m, \; n \in \N$: \\

$\tau \; : \; \R_{m \times n} \to \R_{n \times m} \\\\
\begin{pmatrix}
    a_{11} & a_{12} & a_{13} & \dots  & a_{1n} \\
    a_{21} & a_{22} & a_{23} & \dots  & a_{2n} \\
    \vdots & \vdots & \vdots & \ddots & \vdots \\
    a_{m1} & a_{m2} & a_{m3} & \dots  & a_{mn}
\end{pmatrix} \mapsto \begin{pmatrix}
    a_{11} & a_{21} & a_{31} & \dots  & a_{m1} \\
    a_{12} & a_{22} & a_{32} & \dots  & a_{m2} \\
    \vdots & \vdots & \vdots & \ddots & \vdots \\
    a_{1n} & a_{2n} & a_{3n} & \dots  & a_{mn}
\end{pmatrix}$ \\\\

Тоест $\tau$ е функция, която от една матрица констуира матрица, чиито стълбове съвпадат с редовете на подадената матрица. \\

Нека резултатната матрица от прилагането на функцията $\tau$ към матрицата $A$ означим с $A^t$, тоест $A^t = \tau(A)$.
Матрицата $A^t$ ще наричаме транспонирана матрица на матрицата $A$, а самото действие ще наричаме транспониране. 

\subsubsection*{Примери:}

$
\begin{pmatrix}
    1 & 2 & 3 \\
    0 & 5 & 7
\end{pmatrix}^t = \begin{pmatrix}
    1 & 0 \\
    2 & 5 \\
    3 & 7
\end{pmatrix} \\\\\\
\begin{pmatrix}
    1 & 2 & 3 & -1 \\
    0 & 5 & 7 & -1 \\
    -4 & 2 & 5 & 1
\end{pmatrix}^t = \begin{pmatrix}
    1 & 0 & -4 \\
    2 & 5 & 2 \\
    3 & 7 & 5 \\
    -1 & -1 & 1
\end{pmatrix} \\\\\\
\begin{pmatrix}
    3 & -2 \\
    4 & 5
\end{pmatrix}^t = \begin{pmatrix}
    3 & 4 \\
    -2 & 5
\end{pmatrix}
$

\subsection*{Умножение на матрици}

Две матрици ще умножаваме по следното правило за всеки ред от лявата матрица и всеки стълб от дясната матрица
сумираме произведението на елементите на еднакви позиции. Следователно броя на колоните на лявата и редовете на дясната матрица трябва да съвпадат ... \\\\

Нека $A \in \R_{m \times n}, \; B \in \R_{n \times k}$ и нека $C = A.B \implies C \in \R_{m \times k}$. \\

Нека матриците $A$, $B$ и $C$ съкратено са записани по следния начин: \\

$A = (a_{ij})_{m \times n}, \; B = (b_{hl})_{n \times k}$ и нека $C = (c_{st})_{m \times k}$ тогава за елементите на матрицата $C$ е изпълнено: \\

За всяко $s = 1, \; \dots, \; m, \; t = 1, \; \dots, \; k \; c_{st} = a_{s1}.b_{1t} + a_{s2}.b_{2t} + \dots + a_{sn}.b_{nt} = \displaystyle\sum_{v = 1}^n (a_{sv}.b_{vt}) $  \\\\

\subsubsection*{Примери:}

$
\begin{pmatrix}
    1 & 2 & 3 \\
    0 & 5 & 7
\end{pmatrix} . \begin{pmatrix}
    1 & 2 & 3 & -1 \\
    0 & 5 & 7 & -1 \\
    -4 & 2 & 5 & 1
\end{pmatrix} = \\\\\\
= \begin{pmatrix}
    1.1 + 2.0 + 3.(-4) & 1.2 + 2.5 + 3.2 & 1.3 + 2.7 + 3.5 & 1.(-1) + 2.(-1) + 3.1 \\
    0.1 + 5.0 + 7.(-4) & 0.2 + 5.5 + 7.2 & 0.3 + 5.7 + 7.5 & 0.(-1) + 5.(-1) + 7.1
\end{pmatrix} = \\\\\\
= \begin{pmatrix}
    1 + 0 - 12 & 2 + 10 + 6 & 3 + 14 + 15 & -1 -2 + 3 \\
    0 + 0 - 28 & 0 + 25 + 14 & 0 + 35 + 35 & 0 - 5 + 7
\end{pmatrix} = \begin{pmatrix}
    -11 & 18 & 32 & 0 \\
    -28 & 39 & 70 & 2
\end{pmatrix} \\\\\\
\begin{pmatrix}
    1 & 2 & 3 \\
    0 & 5 & 7
\end{pmatrix} . \begin{pmatrix}
    0 \\
    1 \\
    -1
\end{pmatrix} = \begin{pmatrix}
    1.0 + 2.1 + 3.(-1) \\
    0.0 + 5.1 + 7.(-1)
\end{pmatrix} = \begin{pmatrix}
    -1 \\
    -2
\end{pmatrix} \\\\\\
\begin{pmatrix}
    1 & -1 \\
    1 & 0
\end{pmatrix} . \begin{pmatrix}
    3 & -2 \\
    4 & 5
\end{pmatrix} = \begin{pmatrix}
    1.3 + (-1).4 & 1.(-2) + (-1).5 \\
    1.3 + 0.4 & 1.(-2) + 0.5
\end{pmatrix}  = \begin{pmatrix}
    -1 & -7 \\
    3 & -2
\end{pmatrix} \\\\\\
\begin{pmatrix}
    3 & -2 \\
    4 & 5
\end{pmatrix} . \begin{pmatrix}
    1 & -1 \\
    1 & 0
\end{pmatrix} = \begin{pmatrix}
    3.1 + (-2).1 & 3.(-1) + (-2).0 \\
    4.1 + 5.1 & 4.(-1) + 5.0
\end{pmatrix} = \begin{pmatrix}
    1 & -3 \\
    9 & -4
\end{pmatrix}
$

\section*{Скаларно произведение на два вектора от $\R^n$}

Очевидно на всеки вектор от множеството $\R^n$ директно можем да съпоставим единствена матрица от $\R_{1 \times n}$ по следното правило: \\

$\psi \; : \; \R^n \to \R_{1 \times n} \\
(a_1, \; a_2, \; \dots, \; a_n) \mapsto \begin{pmatrix}
    a_1 & a_2 & \dots & a_n
\end{pmatrix}$ \\

Очевидно по аналогичен начин на всеки вектор от множеството $\R^n$ директно можем да съпоставим единствена матрица от $\R_{n \times 1}$ по следното правило: \\

$\Psi \; : \; \R^n \to \R_{n \times 1} \\
(a_1, \; a_2, \; \dots, \; a_n) \mapsto \begin{pmatrix}
    a_1 \\
    a_2 \\
    \vdots \\
    a_n
\end{pmatrix} $ \\

Под скаларно произведение на два вектора ние ще разбираме следната функция: \\

$\sigma \; : \; \R^n \times \R^n \to \R \\
\sigma(a, \; b) = \sigma\left((a_1, \; a_2, \; \dots, \; a_n) , \; (b_1, \; b_2, \; \dots, \; b_n) \right) = \\
= \psi((a_1, \; a_2, \; \dots, \; a_n)).\Psi((b_1, \; b_2, \; \dots, \; b_n)) = \\\\
= \begin{pmatrix}
    a_1 & a_2 & \dots & a_n
\end{pmatrix}.\begin{pmatrix}
    b_1 \\
    b_2 \\
    \vdots \\
    b_n
\end{pmatrix} = \displaystyle\sum_{i = 1}^n (a_i.b_i)$ \\\\

Въвеждаме и следния "съкратен" \; запис за скаларно произведение на два вектора: \\\\
$<a, \; b> = \sigma(a, \; b) = \sigma\left((a_1, \; a_2, \; \dots, \; a_n) , \; (b_1, \; b_2, \; \dots, \; b_n) \right) = \displaystyle\sum_{i = 1}^n (a_i.b_i)$

\subsubsection*{Примери:}

$<(2), \; (3)> = 2.3 = 6$ \\

$<(2, \; 3), \; (3, \; 4)> = 2.3 + 3.4 = 6 + 12 = 18$ \\

$<(2, \; 3, \; 0, \; -1), \; (3, \; 4, \; 1, \; 0)> = 2.3 + 3.4 + 0.1 + (-1).0 = 6 + 12 + 0 + 0 = 18$ \\

Да си припомним, че дефинирахме дължина на вектор $a$ от $\R^n$ по следния начин: \\

$\|a\| = \|(a_1, \; a_2, \; \dots, \; a_n)\| = \displaystyle\sqrt{\displaystyle\sum_{i = 1}^n a_i^2}$ \\\\

Тогава е ясно и че $\|a\| = \sqrt{<a, a>}$

\section*{Противополжна матрица}

Нека $m, \; n \in \N$ са две фиксирани естесвени числа. Нека $A \in \R_{m \times n}$ е произволна матрица. \\

Противоположната матрицата на $A$ ще означаваме с $-A$ и тя е матрица, която има следното свойство: \\\\

$A + (-A) = -A + A = \begin{pmatrix}
    0 & 0 & 0 & \dots  & 0 \\
    0 & 0 & 0 & \dots  & 0 \\
    \vdots & \vdots & \vdots & \ddots & \vdots \\
    0 & 0 & 0 & \dots  & 0
\end{pmatrix}$ \\\\

Следователно е ясно, че $-A = (-1).A$.

\end{document}