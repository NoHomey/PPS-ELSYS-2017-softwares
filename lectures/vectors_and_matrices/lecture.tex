\documentclass[12pt]{article}

\usepackage[left=3cm,right=3cm,top=1cm,bottom=2cm]{geometry}
\usepackage{amsmath,amsthm}
\usepackage{amssymb}
\usepackage{tikz}
\usepackage{lipsum}
\usepackage[T1,T2A]{fontenc}
\usepackage[utf8]{inputenc}
\usepackage[bulgarian]{babel}
\usepackage[normalem]{ulem}

\newcommand{\N}{\mathbb{N}}
\newcommand{\R}{\mathbb{R}}

\setlength{\parindent}{0mm}
    
\title{Вектори и матрици}
\author{Иво Стратев}
    
\begin{document}
\maketitle

\section*{Вектори}

\subsection*{Наредена двойка от реални числа}

Произволна наредена двойка от реални числа записваме като: $(a, \; b)$, където $a, \; b \in \R$. \\

Множеството от всички наредени двойки от реални числа означаваме с $\R^2$ и на езика на математиакта записваме като:
$\R^2 = \{(a, \; b) \; | \; a, \; b \in \R\}$ \\

Подобно на начина, по който си мислим за реални числа като точки от една безкрайна права множеството $\R^2$ можем
да си мислим че е множеството от точки в равнината. По този начин на наредената двойка $(1, \; 2)$ можем да съпоставим
точка от равнината с координати $(1, \; 2)$, която се намира на разтояние $1$ спрямо центъра на координаттната система
по абцисата и на разтояние $2$ по ординатата . Графично: \\\\

\begin{tikzpicture}
    \draw[thin,gray!40] (-4,-4) grid (4,4);
    \draw[<->] (-4,0)--(4,0) node[right]{$x$};
    \draw[<->] (0,-4)--(0,4) node[above]{$y$};
    \draw[fill] (1, 2) circle (2pt) node[anchor=south west]{$(1, \; 2)$};
  \end{tikzpicture}

\subsection*{Наредена тройка от реални числа}
  
Аналогично на понятието наредена двойка от реални числа можем да си въведем понятието наредена тройка от реални числа.

Произволна наредена тройка от реални числа записваме като: $(a, \; b, \; c)$, където $a, \; b, \; c \in \R$. \\
  
Множеството от всички наредени тройки от реални числа означаваме с $\R^3$ и на езика на математиакта записваме като:
$\R^3 = \{(a, \; b, \; c) \; | \; a, \; b, \; c \in \R\}$ \\
  
Подобно на начина, по който си мислим за наредените двойки от реални числа катоточки в равнината.
Можем да си мислим за една наредена тройка от наредени реални числа като точка от пространството.
Така на произволна наредена тройка $(x, \; y, \; z)$ от реални числа можем по единствен начин да съпоставим
точка от пространството, която се намира на отместване $x$ спрямо началото на координатната система по абсцисата (остта $Ox$),
отместване $y$ спрямо началото на координатната система по ординатата (остта $Oy$) и на отместване $z$ по апликатата (остта $Oz$).

\subsection*{Наредена n-торка от реални числа}

Понятието наредена n-торка от реални числа можем да изградим като обобщим това за наредена двойка, тройка. \\

Произволна наредена n-торка от реални числа записваме като: $(a_1, \; a_2, \; \dots, \; a_n)$, където $a_1, \; a_2, \; \dots, \; a_n \in \R$. \\

Аналогично множеството от всички наредени n-торки от реални числа ще обозначаваме с $\R^n$ и
$\R^n = \{(a_1, \; a_2, \; \dots, \; a_n) \; | \; a_1, \; a_2, \; \dots, \; a_n \in \R\}$. \\

При $n > 3$ вече нямаме пряка геометрична репрезентация ... \\

\subsection*{Вектор}

Нека $n$ е фиксирано естествено число (тоест $n \in \N$). \\\\

Сега ако в множеството $\R^n$ си дефинираме по подходящ начин операция събиране на наредени n-торки и умножение с реално число на наредена n-торка получаваме нещо, на което в математиката (а и не само там)
наричаме вектор(и). Тези две операции дефинираме по следния начин: \\

Нека $a = (a_1, \; a_2, \; \dots, \; a_n)$ и $b = (b_1, \; b_2, \; \dots, \; b_n)$ са произволни наредени n-торки от реални числа. Тогава
$a + b \overset{def}{=} (a_1 + b_1, \; a_2 + b_2, \; \dots, \; a_n + b_n)$. Тоест извършваме покомпонентно операцията събиране на реални числа за всяка компонента. \\

Нека $a = (a_1, \; a_2, \; \dots, \; a_n)$ е произволна наредена n-торка от реални числа и $\lambda$ е произволно реално число. Тогава
$\lambda.a \overset{def}{=} (\lambda.a_1, \; \lambda.a_2, \; \dots, \; \lambda.a_n)$. Отново извършваме покомпонентно операцията умножение на реални числа за всяка компонента. \\

Под вектор ние ще разбираме всяка наредена n-торка от реални числа имайки в предвид двете операции, които дефинирахме.\\

Забележка: по начина по-който дефинирахме операцията умножение с число е ясно, че можем да я прилагаме върху каквато и да е
наредена k-торка (тоес и за наредена двойка, тройка, четворка и тн.). Обаче операцията събиране можем да прилагаме само върху
вектори от един и същ тип (тоест не можем директно да съберем наредена двойка с тройка). \\

Забележка: честно ще изпускаме да пишем точката за умножение подобно на изпускането ѝ което сте свикнали да правите в часовете по математика. \\

Забележка: понятието вектор в математика представлява абстракция която има широко приложение и различни математически обекти се превръщат във вектори с подходяща
дефиниция на двете операции. В този курс ще се занимаваме с доста ограничени видове вектори, които обаче се приемат за фундаментални и основни и напрактика почти
всички вектори са подобни на тях.

\subsubsection*{Геометрична интерпретация на двете операции в множеството $\R^2$}

Геометрично векторите ще изобразяваме като вместо точка свържем центъра на координатната система и за край изпозлваме стрелка в съответната посока. Тоест:\\\\

\begin{tikzpicture}
    \draw[thin,gray!40] (-4,-4) grid (4,4);
    \draw[<->] (-4,0)--(4,0) node[right]{$x$};
    \draw[<->] (0,-4)--(0,4) node[above]{$y$};
    \draw[fill] (1, 2) circle (2pt) node[anchor=south west]{$(1, \; 2)$};
    \draw[line width=2pt,green,-stealth](0,0)--(1,2);
\end{tikzpicture} \\\\
  
Нека сега съберем векторите: \\

\begin{tikzpicture}
    \draw[thin,gray!40] (-5,-5) grid (5,5);
    \draw[<->] (-5,0)--(5,0) node[right]{$x$};
    \draw[<->] (0,-5)--(0,5) node[above]{$y$};
    \draw[line width=2pt,blue,-stealth](0,0)--(1,3) node[anchor=south west]{$\boldsymbol{(1,3)}$};
    \draw[line width=2pt,red,-stealth](0,0)--(2,1) node[anchor=south west]{$\boldsymbol{(2,1)}$};
    \draw[line width=2pt,purple,-stealth](0,0)--(3,4) node[anchor=south west]{$\boldsymbol{(3,4)}$};
\end{tikzpicture} \\\\

Казваме, че събираме векторите е по така нареченото правило на триъгълника.
Тоест като в края на първия успоредно пренесем втория такаче началото на пренесения на съвпада с края на другия. Или: \\

\begin{tikzpicture}
    \draw[thin,gray!40] (-5,-5) grid (5,5);
    \draw[<->] (-5,0)--(5,0) node[right]{$x$};
    \draw[<->] (0,-5)--(0,5) node[above]{$y$};
    \draw[line width=2pt,blue,-stealth](0,0)--(1,3) node[anchor=south west]{$\boldsymbol{(1,3)}$};
    \draw[line width=2pt,red,-stealth](0,0)--(2,1) node[anchor=south west]{$\boldsymbol{(2,1)}$};
    \draw[line width=2pt,purple,-stealth](0,0)--(3,4) node[anchor=south west]{$\boldsymbol{(3,4)}$};
    \draw[line width=2pt,blue,dashed,-stealth](2,1)--(3,4);
\end{tikzpicture} \\\\

Или казваме, че събираме вектори по правилото на успоредника, защото всъщност сборът на двата вектора е диагонал
на успоредник построен със "страни" двата вектора, или: \\

\begin{tikzpicture}
    \draw[thin,gray!40] (-5,-5) grid (5,5);
    \draw[<->] (-5,0)--(5,0) node[right]{$x$};
    \draw[<->] (0,-5)--(0,5) node[above]{$y$};
    \draw[line width=2pt,blue,-stealth](0,0)--(1,3) node[anchor=south east]{$\boldsymbol{(1,3)}$};
    \draw[line width=2pt,red,-stealth](0,0)--(2,1) node[anchor=south west]{$\boldsymbol{(2,1)}$};
    \draw[line width=2pt,purple,-stealth](0,0)--(3,4) node[anchor=north west]{$\boldsymbol{(3,4)}$};
    \draw[line width=2pt,blue,dashed,-stealth](2,1)--(3,4);
    \draw[line width=2pt,red,dashed,-stealth](1,3)--(3,4);
\end{tikzpicture} \\\\

Нека сега да умножим вектора $(1, \; 2)$ с числото $2$ и с числото $-1$.\\

\begin{tikzpicture}
    \draw[thin,gray!40] (-5,-5) grid (5,5);
    \draw[<->] (-5,0)--(5,0) node[right]{$x$};
    \draw[<->] (0,-5)--(0,5) node[above]{$y$};
    \draw[line width=2pt,blue,-stealth](0,0)--(1,2) node[anchor=south east]{$\boldsymbol{(1,2)}$};
    \draw[line width=2pt,red,-stealth](0,0)--(-1,-2) node[anchor=south east]{$\boldsymbol{(-1,-2)}$};
    \draw[line width=2pt,green,dashed,-stealth](0,0)--(2,4) node[anchor=south east]{$\boldsymbol{(2,4)}$};
\end{tikzpicture} \\\\

Очевидно всеки вектор геометрически се характеризира с посока и дължина. Сега ще получим формула за дължината на даден вектор,
която след това ще обобщим за n-мерните вектори (наредните n-торки). \\

Нека си начертаем вектора $(1, \; 2)$, след което успоредно си пренесем разтоянието по ординатата в края на разтоянието по абсцисата: \\

\begin{tikzpicture}
    \draw[thin,gray!40] (-4,-4) grid (4,4);
    \draw[<->] (-4,0)--(4,0) node[right]{$x$};
    \draw[<->] (0,-4)--(0,4) node[above]{$y$};
    \draw[line width=2pt,blue,-stealth](0,0)--(1,2) node[anchor=south east]{$\boldsymbol{(1,2)}$};
    \draw[line width=1pt,blue,dashed,-stealth](1,0)--(1,2);
    \draw[line width=1pt,blue,dashed,-stealth](0,0)--(1,0);
\end{tikzpicture} \\\\

Очевидно получаваме правоъгален триъгълник с дължини на катетите $1$ и $2$ тогава от формулата на Питагор
за хипотенузата на този тригълник $\rho$ получаваме: \\
$\rho^2 = 1^2 + 2^2$, но понеже дължината винаги е неотриателно число получаваме: \\
$\rho = \sqrt{1^2 + 2^2} = \sqrt{5}$. \\

Нека сега сметнем дължината $h$ на вектора $2(1, \; 2) = (2, \; 4)$: \\
$h^2 = \sqrt{2^2 + 4^2} = \sqrt{2^2.1^2 + 2^2.2.^2} = \sqrt{2^2.(1^2 + 2^2)} = 2\sqrt{5}$ \\

Ако $a = (a_1, \; a_2)$ дължината на вектора $a$ ще означаваме с $\|a\| = \sqrt{a_1^2 + a_2^2}$ \\

Сега ще си докажем едно много важно свойстово на умножението на вектор с число: \\

Нека $a = (x, \; y) \in \R^2$ и $\lambda \in \R$ тогава $\|\lambda . a\| = |\lambda|.\|a\|$ \\

Доказателство: $\|\lambda . a\| = \sqrt{(\lambda x)^2 + (\lambda y)^2} = \sqrt{\lambda^2(x^2 + y^2)} = \\
= \sqrt{\lambda^2}\sqrt{x^2 + y^2} = |\lambda|.\|a\| \qed$ \\

Очевидно посоката на даден вектор можем да определим на базата ъгъла, който той сключва с абсцисата.
На помощ ни идват любимите на учениците тригонометрични функции $\sin(x)$ и $\cos(x)$.
Ще използваме горния чертеж в комбинация със стандартните означения за триъгълници: \\

\begin{tikzpicture}
    \draw[thin,gray!40] (-4,-4) grid (4,4);
    \draw[<->] (-4,0)--(4,0) node[right]{$x$};
    \draw[<->] (0,-4)--(0,4) node[above]{$y$};
    \draw[line width=2pt,blue,-stealth](0,0)--(1,2) node[anchor=south east]{$\boldsymbol{(1,2)}$};
    \draw[line width=1pt,blue,dashed,-stealth](1,0)--(1,2);
    \draw[line width=1pt,blue,dashed,-stealth](0,0)--(1,0);
    \draw (0.5, 0) node[anchor=north]{$a$};
    \draw (1, 1) node[anchor=west]{$b$};
    \draw (0.5, 1) node[anchor=south east]{$c$};
    \draw (0.1, 0) node[anchor=south west]{$\varphi$};
\end{tikzpicture} \\\\

Очевидно $a = 1, \; b = 2, \; c = \|(1, \; 2)\| = \sqrt{5}$ \\

Така $ \sin(\varphi) \overset{def}{=} \frac{b}{c} = \frac{2\sqrt{5}}{5} $ и $ \cos(\varphi) \overset{def}{=} \frac{a}{c} = \frac{\sqrt{5}}{5} $. Нека сега се уверим,
че умножавайки вектора $(1, \; 2)$ с произволно положително реално число реално резултатния вектор ще има същата посока.
Нека $\mu \in \R$ и $\mu > 0$ тогава за $\mu(1, \; 2) = (\mu, \; 2\mu) $ имаме
$ \sin(\psi) \overset{def}{=} \frac{\mu.b}{\mu.c} = \frac{\mu2\sqrt{5}}{\mu5} = \frac{2\sqrt{5}}{5} = \sin(\varphi) $
и $ \cos(\psi) \overset{def}{=} \frac{\mu.a}{\mu.c} = \frac{\mu\sqrt{5}}{\mu5} = \frac{\sqrt{5}}{5} = \cos(\varphi) $

Нека сега $\mu < 0 \implies |\mu| = -\mu \implies \mu = -|\mu| $. \\
Тогава за $\mu(1, \; 2) = (\mu, \; 2\mu) $ имаме: \\
$ \sin(\psi) \overset{def}{=} \frac{-|\mu|.b}{|\mu|.c} = \frac{-|\mu|2\sqrt{5}}{|\mu|5} = -\frac{2\sqrt{5}}{5} = -\sin(\varphi) $ \\
и $ \cos(\psi) \overset{def}{=} \frac{-|\mu|.a}{|\mu|.c} = \frac{-|\mu|\sqrt{5}}{|\mu|5} = -\frac{\sqrt{5}}{5} = -\cos(\varphi) $ \\

Очевидно умножавайки вектор с отрицателно реално число получаваме вектор с противоположна посока. \\

Ето защо когато умножаваме даден вектор с число още казваме, че го умножаваме със скалар или че скалираме дадения вектор.
Защото по този начин променяме само дължината на дадения вектор, но неговата посока се запазва ако умножаваме с положително по знак число и
получаваме противоположна посока в случай на отрицателно число. \\

Знаем, че модула на реално число играе ролята на разтояние от центъра на реалната права, токчата отговаряща на числото $0$.
Също така знаем, че ако $\lambda \in \R$ то $|\lambda| = \sqrt{\lambda^2}$ И има точно две реални числа на разтояние
$|\lambda|$ - $\lambda$ и $-\lambda$, защото върху една права имаме точно две посоки положителна и отрицателна. \\

В равнината обаче имаме безброй много посоки защото имаме безборй много ъгли, които дават различни двойка стойности $(\sin(\varphi), \; \cos(\varphi))$,
по-точно $\varphi \in [0, \pi)$, но това е отворен интервал от реални числа и с помощта на математиката (логиката и теорията на множествата) лесно
се доказва, че интервала $[0, \pi)$ съдържа безброй много реални числа (всъщност той съдържа толкова на брой реални числа колкото и самото множество $\R$). \\

Така получаваме цяла окръжност от точки (вектори), които са на едно и също разстояние: \\

\begin{tikzpicture}
    \draw[thin,gray!40] (-4,-4) grid (4,4);
    \draw[<->] (-4,0)--(4,0) node[right]{$x$};
    \draw[<->] (0,-4)--(0,4) node[above]{$y$};
    \draw[line width=2pt,blue,-stealth](0,0)--(1,2) node[anchor=south west]{$\boldsymbol{(1,2)}$};
    \draw[line width=2pt,blue,dashed,-stealth](0,0) circle ({sqrt(5)});
\end{tikzpicture}

\subsubsection*{Дефиниция на дължина на вектор}

Дължина и разстояние за нас практически са едно и също понятие и така
дължина на едномерен вектор всъщност съвпада с модула на числото, коеот е единствена компонента на вектора, тоест:\\

$\|(a)\| = \sqrt{a^2} = |a|$. \\

Двумерния случай вече е ясен, дължина там е мярката на радиоса на окръжността, която можем да опишем около даден вектор. \\

В тримерния случай по аналогия на идеята с окръжността получаваме, че дължина на тримерен вектор всъщност
ще съвпада с радиоса на единствената сфера, която можем да опишем около дадения вектор, тоест: \\

$\|(a, \; b, \; c)\| = \sqrt{a^2 + b^2 + c^2}$. \\

Така за n-мерен вектор е естесвено да стигнем до следната формула: \\

$\|(a_1, \; a_2, \; \dots, \; a_n)\| = \sqrt{a_1^2 + a_2^2 + \dots + a_n^2} = \sqrt{\displaystyle\sum_{i = 1}^n a_i^2}$ \\

Очевидно отново ще имаме свойстовото $\|\lambda.(a_1, \; a_2, \; \dots, \; a_n)\| = |\lambda|.\|(a_1, \; a_2, \; \dots, \; a_n)\|$



\end{document}