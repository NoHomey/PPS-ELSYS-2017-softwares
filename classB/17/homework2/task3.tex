\documentclass{article}
\usepackage[utf8]{inputenc}
\usepackage{amsmath}
\usepackage{amssymb}
\usepackage[T1,T2A]{fontenc}
\usepackage[bulgarian]{babel}
\author{Martin Datsev}
\title{Homework 2 - Task 3}
% Start the document
\begin{document}
\maketitle
\section*{Твърдение:}
Нека $n$ е фиксирано произволно естествено число.
Тогава за произволни три вектора $a$, $b$ и $c$ от $\mathbb{R}^n$ е изпълнено равенството:

\[<a + b, \; c> = <a, \; c> + <b, \; c>\]
\subsection*{Доказателство:}


\[<a, \; b> = \displaystyle\sum_{i = 1}^n (a_i.b_i) \tag{1} \label{eq:1}\]

Ако
\[a = (a_1, \; a_2, \; \dots, \; a_n), \; b = (b_1, \; b_2, \; \dots, \; b_n),\]
то
\[a + b = (a_1 + b_1, \; a_2 + b_2, \; \dots, \; a_n + b_n) \tag{2} \label{eq:2}\]

От \eqref{eq:1} и \eqref{eq:2}:
\[\implies<a + b, \; c>=\displaystyle\sum_{i = 1}^n ((a_i + b_i).c_i) \tag{3} \label{eq:3}\]

\[<a, \; c> = \displaystyle\sum_{i = 1}^n (a_i.c_i) \tag{4} \label{eq:4}\]

\[<b, \; c> = \displaystyle\sum_{i = 1}^n (b_i.c_i) \tag{5} \label{eq:5}\]

От \eqref{eq:4} и \eqref{eq:5}:
\[\implies<a, \; c> + <b, \; c> = \displaystyle\sum_{i = 1}^n (a_i.c_i) + \displaystyle\sum_{i = 1}^n (b_i.c_i) = \displaystyle\sum_{i = 1}^n (a_i.c_i + b_i.c_i) =\]
\[= \displaystyle\sum_{i = 1}^n ((a_i + b_i).c_i) \overset{\eqref{eq:3}}{=} <a + b, \; c>\]

\end{document}